\documentclass{article}
\usepackage{amsmath}
\usepackage{amssymb}
\usepackage{amsthm}
\usepackage{graphicx}
\usepackage{mathrsfs,mathtools}
\usepackage{bm}

\usepackage{fullpage}
\newtheorem{theorem}{Theorem}
\newtheorem{lemma}[theorem]{Lemma}
\newtheorem{proposition}[theorem]{Proposition}
\newtheorem{corollary}[theorem]{Corollary}
\theoremstyle{definition}
\newtheorem{definition}{Definition}
\usepackage[colorlinks=true,pagebackref,linkcolor=magenta]{hyperref}
\usepackage[colon,sort&compress]{natbib}
%\numberwithin{equation}{section}
\renewcommand\arraystretch{1.2}
\let\underbrace\LaTeXunderbrace
\let\overbrace\LaTeXoverbrace
\newcommand{\argmax}{\operatornamewithlimits{argmax}}
\newcommand{\argmin}{\operatornamewithlimits{argmin}}
\renewcommand{\tilde}{\widetilde}
\renewcommand{\Pr}{\mathbb{P}}
\renewcommand{\Re}{\mathbb{R}}
\bibliographystyle{plainnat}
\begin{document}
\title{PyGraphStat}
\author{$\sigma$(Vogelstein, Sussman)}

\maketitle

\section{Overview} 
This package provides the ability to perform statistical inference on graphs. We provide methods to perform hypothesis testing, graph classification, 
unsupervised vertex clustering and semi-supervised vertex clustering. We also provide embedding techniques that give a representation of the graph in Euclidean space. Additionally we provide methods to generate random graphs such as stochastic blockmodel graphs and random dot product graphs.

We use networkx Python package's built in classes to handle graphs as well as the tools to 

\section{Random Graph Generation}
We seek to implement fast generation of random graphs for models of interest for studying statistical inference on graph. These graphs include 
\begin{enumerate}
  \item Stochastic Blockmodel Graphs \citep{Holland}(Holland)
  \item Random Dot Product Graphs \citep{Scheinerman}(Scheinerman)
  \item Colored Random Graphs \citep{Marchette}(Marchette)
  \item Errorfully Obeserved Grahps \citep{PriebeBock}
\end{enumerate}

\section{Graph Embedding}

\section{Hypothesis Testing}

\section{Vertex Clustering}

\section{Vertex Nomination}

\section{Estimation}


\section{Graph Classification}



\end{document}	
